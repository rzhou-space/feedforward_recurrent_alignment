\documentclass[11pt]{article}
\usepackage[ngerman]{babel}
\usepackage[utf8]{inputenc}
\usepackage[T1]{fontenc}
\usepackage{float}
\usepackage{lmodern,amsmath,amssymb,amstext,amsfonts,mathrsfs,graphicx,caption, subcaption}
\usepackage[width=14cm]{geometry}
\usepackage[colorlinks,pdfpagelabels,pdfstartview = FitV,bookmarksnumbered = true, bookmarksopenlevel=section, linkcolor = black,hypertexnames = false,citecolor = black,pdfpagelabels=false]{hyperref}
\usepackage{tablefootnote}
%\usepackage{rotating}
\usepackage{textcmds, enumitem}
\usepackage{sidecap} %, indentfirst
\usepackage[labelfont={bf,sf},font={small},labelsep=space]{caption}
\usepackage{chngcntr} % 			** Damit die Bilder Tabellen und Gleichungen 
\counterwithin{figure}{section}	%	** alle nach Kapiteln nummeriert sind.
\counterwithin{table}{section}%		**
\counterwithin{equation}{section}%	**	

\begin{document}
	\begin{titlepage}

	\section*{Zusammenfassung}
	
	Die theoretische Neurowissenschaft trägt dazu bei, die zugrundeliegenden Mechanismen neuronaler Systeme durch theoretische Modellierung und Analyse zu verstehen. Die Dynamik der kortikalen Netzwerke lässt sich mittels feedforward-rekurrenter Netzwerke modellieren. Wenn neuartige visuelle Eingaben durch feedforward-Eingaben repräsentiert werden, kann die Koordination zwischen dem feedforward- und dem rekurrenten Netzwerk die Zuverlässigkeit der finalen Repräsentationen der Antwort bestimmen. Dies kann dazu beitragen, die Mechanismen zu erforschen, die es endogen generierten Netzwerken ermöglichen, reife Repräsentationen mit dem Einsetzen sensorischer Erfahrungen zu bilden. Bisher beschränkt sich die konzeptuelle Modellierung jedoch auf den Fall idealisierter symmetrischer neuronaler Interaktionen. Hier erweitern wir die bisherige Modellierung um allgemeinere Interaktionsstrukturen der Netzwerke und untersuchen die Leistung der Modellierung unter komplexeren Bedingungen.
	Für asymmetrische neuronale Interaktionen kann die Symmetrisierung die meisten Musterinformationen bewahren und eine Anordnung in der Realzahlenebene ermöglichen. Das durch Weißrauschen hervorgerufene Aktivitätsmuster ist ein möglicher Kandidat für eine Anordnung, um unbekannte rekurrente Interaktionen anzunähern. Die Einbindung des Hebb'schen Lernens in das Modell zeigt das Potenzial der Plastizität zur Erforschung der Mechanismen. Die Arbeit demonstriert die verschiedenen Modifikationen für verschiedene Netzwerkstrukturbedingungen, die die feedforward-rekurrente Koordination ermöglichen.
	Die theoretischen Untersuchung der Koordination zwischen feedforward- und rekurrenten Netzwerken unter verschiedenen Interaktionsbedingungen können die bisherigen Modellierungsansätze verallgemeinern. Darüber hinaus können die Ergebnisse neue Perspektiven bieten, um das Verständnis der Mechanismen für durch Erfahrung gesteuerte Entwicklungen in kortikalen Netzwerken zu vertiefen.
	
	\end{titlepage}
	
\end{document}