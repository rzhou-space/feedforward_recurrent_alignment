\documentclass[11pt]{article}
\usepackage[english]{babel}
\usepackage[utf8]{inputenc}
\usepackage[T1]{fontenc}
\usepackage{float}
\usepackage{lmodern,amsmath,amssymb,amstext,amsfonts,mathrsfs,graphicx,caption, subcaption}
\usepackage[width=14cm]{geometry}
\usepackage[colorlinks,pdfpagelabels,pdfstartview = FitV,bookmarksnumbered = true, bookmarksopenlevel=section, linkcolor = black,hypertexnames = false,citecolor = black,pdfpagelabels=false]{hyperref}
\usepackage{tablefootnote}
%\usepackage{rotating}
\usepackage{textcmds, enumitem}
\usepackage{sidecap} %, indentfirst
\usepackage[labelfont={bf,sf},font={small},labelsep=space]{caption}
\usepackage{chngcntr} % 			** Damit die Bilder Tabellen und Gleichungen 
\counterwithin{figure}{section}	%	** alle nach Kapiteln nummeriert sind.
\counterwithin{table}{section}%		**
\counterwithin{equation}{section}%	**	
\usepackage{tabularx}

\begin{document} %TODO: Die Liste noch aufräumen und sinnvoll zusammen ordnen!!!!!!!!!!!!!!
	\section{List of Important Symbols}
	
	% $J$ \hspace{2cm} interaction matrix for recurrent network \\
	% $n$ \hspace{2cm} number of neurons in recurrent network \\
	% $R$ \hspace{2cm} radius for eigenvalue distribution \\
	% $\lambda_i$ \hspace{2cm} original eigenvalues of recurrent interaction matrix $J$\\
	
	% \begin{table}[]
		% \fontsize{12pt}{12pt}\selectfont{%
			% \begin{tabular}{ll}
				% $J$ & interaction matrix for recurrent network \\
				% $n$ & number of neurons in recurrent network
				% \end{tabular}%
			% }
		% \end{table}
	
	% Notations.
	\begin{table}[H]
		%\centering
		\begin{tabularx}{\textwidth}{lX}
			$n \in \mathbb{R}$  & number of neurons in recurrent network \\
			
			$J \in \mathbb{R}^{n \times n}$  & interaction matrix for recurrent network\\
			
			$R \in \mathbb{R}$  & radius for eigenvalue distribution \\
			
			%$\lambda_i \in \mathbb{R}$ or $\mathbb{C}$  & original eigenvalues of recurrent interaction matrix $J$\\
			%$\Tilde{\lambda_i} \in \mathbb{R}$ or $\mathbb{C}$  & re-scaled eigenvalues of recurrent interaction matrix $J$ \\
			$h \in \mathbb{R}^{n \times 1}$  & mean firing rates feedforward inputs to recurrent network if not otherwise defined in contexts\\
			
			$r \in \mathbb{R}^{n \times 1}$ & response from recurrent network \\
			
			$\tau_r \in \mathbb{R}$ & time constant for recurrent network dynamic \\
			
			$r^* \in \mathbb{R}^{n \times 1}$ & steady state response from recurrent network determined by (\ref{eq:steady_state}) \\
			
			$I_n \in \mathbb{R}^{n \times n}$ & identity matrix \\
			
			$A \in \mathbb{R}^{n \times n}$ & Jacobian matrix of feedforward recurrent dynamics (\ref{eq:sym_response_ODE}) \\
			
			$E \in \mathbb{R}^{n \times n}$ & matrix containing the eigenvectors of symmetric recurrent interaction matrix $J$ column-wise \\
			
			$\Lambda \in \mathbb{R}^{n \times n}$ or $\mathbb{C}^{n \times n}$ & diagonal matrix containing the eigenvalues for recurrent interaction matrix $J$ on the diagonal \\
			
			$\nu \in \mathbb{R}$ & feedforward recurrent alignment score\\
			
			% $\mu \in \mathbb{R}^{n \times 1}$ & mean vector for multivariate 
			$\sigma_{\text{trial}} \in \mathbb{R}$ & variance constant for trial-to-trial correlation \\
			
			$\beta_s \in \mathbb{R}$ & trial-to-trial correlation defined in rq.(\ref{eq:ttc_sym}) \\
			
			$\sigma_{\text{time}} \in \mathbb{R}$ & variance constant for intra-trial stability \\
			
			$\bar{c}{(\Delta \tilde{t})} \in \mathbb{R}$ & intra-trial stability defined in eq.(\ref{eq:its_sym}) \\
			
			$0_v \in \mathbb{R}^{n \times 1}$ & zero vector in length of number of neurons $n$ \\
			
			$\Sigma \in \mathbb{R}^{n \times n}$ & covariance matrix for input patterns. There are $\Sigma^{\text{Dim}}$, $\Sigma_{\text{spont}}$, and $\Sigma_{\text{Low}}$ for different contexts.\\
			 
			$d_{\text{eff}} \in \mathbb{R}$ & the linear effective dimensionality defined in eq.(\ref{eq:effective_dimensionality_analytical}). $d_{\text{eff, ana}}$ is the analytical formulation and $d_{\text{eff, emp}}$ the empirical for effective dimensionality \\
			
			$\gamma \in \mathbb{R}$ & the alignment of evoked activity to spontaneous activity with elements defined in eq.(\ref{eq:align_to_spont_act_sym}) \\
			
			$a \in \mathbb{R}$ & degree of symmetry in construction of asymmetric RNNs in eq.(\ref{eq:asym_interaction_matrix}) \\
			
			$\tilde{h} \in \mathbb{R}^{n \times 1}$ & feedforward inputs with modifications mentioned in section \ref{sec:modify_ffrec_alignment_score} \\
			
			$N \in \mathbb{R}$ & number of trials \\
			
			$\beta \in \mathbb{R}$ & parameter that defines the dimensionality in construction of covariance matrix for input. In varies contexts, there are $\beta_{\text{dim}}$, $\beta_{\text{spont}}$, and $\beta_{\text{Low}}$ with $\beta_{\text{Low}} < \beta_{\text{dim}} < \beta_{\text{spont}}$.\\
			
			$W \in \mathbb{R}^{n \times 1}$ & feedforward interaction matrix (vector) for feedforward recurrent network in section \ref{sec:Hebb_learn} \\
		\end{tabularx}
	\end{table}
	
	\begin{table}[H]
		%\centering
		\begin{tabularx}{\textwidth}{lX}
			$v^* \in \mathbb{R}^{n \times 1}$ & steady state for feedforward recurrent interaction during Hebbian learning of feedforward interaction \\
			
			$\rho \in \mathbb{R}$ & projection ratio \\
			
		\end{tabularx}
	\end{table}
	
	
	% Notation values.
	------------------------------------------------------------------------------------------\\
	The values that modeling notations take
	\begin{table}[H]
		%\centering
		\begin{tabular}{ll}
			Notations \hspace{0.5cm} & Values \\
			$n$ & 200 \\
			$R$ & 0.85 \\
			$\tau_r$ & 1 \\
			$\sigma_{\text{trial}}$ & 0.05 \\
			$\sigma_{\text{time}}$ & 0.3 \\
			$M_{\text{dim}}$ & 50 \\
			$M_{\text{spont}}$ & 100\\
			$\Delta t$ & 0.1 \\
			$\Delta \tilde{t}$ & 20\\
			$T$ & 120\\
			$N$ & 500 \\
			$\beta_{\text{Low}}$ & 5\\
			$\kappa$ & 5\\
			$\beta_{\text{dim}}$ & 10\\
			$\beta_{\text{spont}}$ & 20\\
			$M_{\text{Low}}$ & 25\\
			$T_{\text{Hebb}}$ & 50 \\
			
		\end{tabular}
	\end{table}
	
\end{document}
