\documentclass[11pt]{article}
\usepackage[english]{babel}
\usepackage[utf8]{inputenc}
\usepackage[T1]{fontenc}
\usepackage{float}
\usepackage{lmodern,amsmath,amssymb,amstext,amsfonts,mathrsfs,graphicx,caption, subcaption}
\usepackage[width=14cm]{geometry}
\usepackage[colorlinks,pdfpagelabels,pdfstartview = FitV,bookmarksnumbered = true, bookmarksopenlevel=section, linkcolor = black,hypertexnames = false,citecolor = black,pdfpagelabels=false]{hyperref}
\usepackage{tablefootnote}
%\usepackage{rotating}
\usepackage{textcmds, enumitem}
\usepackage{sidecap} %, indentfirst
\usepackage[labelfont={bf,sf},font={small},labelsep=space]{caption}
\usepackage{chngcntr} % 			** Damit die Bilder Tabellen und Gleichungen 
\counterwithin{figure}{section}	%	** alle nach Kapiteln nummeriert sind.
\counterwithin{table}{section}%		**
\counterwithin{equation}{section}%	**	

\begin{document}
	\section{Introduction}
	\subsection{Introduction of Feedforward Recurrent Hypothesis}
	% Experimental backgrounds and preprint contents.
	Cortical circuits embody remarkably reliable neural representations of sensory stimuli that are critical for perception and action. 
	Cortical circuits were thought to emerge from a developmental sequence that includes two distinct phases: an early period prior to the onset of experience during which endogenous mechanisms are thought to formulate the initial and a subsequent period during which these early networks are refined 
	under the influence of experience\cite{ackman2014role, feldheim2010visual, goodhill2016can, huberman2008mechanisms}, and a subsequent period during which these early networks are refined under the influence of experience \cite{avitan2018code, barlow1975visual, espinosa2012development, fregnac1984development, white2007vision}. 
	
	The fundamental structure of cortical network representations is thought to arise early in development prior to the onset of sensory experience. However, how these endogenously generated networks respond to the onset of sensory experience, and the extent to which they reorganize with experience remains unclear \cite{tragenap2023nature}. 
	
	In an earlier work from the lab of M.Kaschube and D.Fitzpatrick \cite{dayan2005theoretical}, they focused on the problem of "nature-nurture transform". They tried to clarify the understanding of the capacity of the endogenous cortical network to reliably represent stimulus orientation at the onset of visual experience, and the degree to which visual experience alters endogenous network structure to achieve mature stimulus representations.

	
	To explore these early developmental dynamics, they applied data from the visual cortex of the newborn ferrets obtained through chronic \textit{in vivo} calcium imaging \cite{tragenap2023nature}. Visually evoked activity in visual cortex of postnatal ferrets prior to and following the onset of visual experience were employed. The visual cortex of higher mammals has served as a powerful model for exploring the contributions of these different phases to the development of mature cortical networks. Prior to the onset of visual experience, activity independent mechanisms combine with activity dependent mechanisms driven by patterns of endogenous activity derived from the retina and the LGN \cite{feller1996requirement, meister1991synchronous, penn1998competition} to generate a robust modular network structure in visual cortex that is evident in patterns of spontaneous activity \cite{chiu2001spontaneous, smith2018distributed}. This endogenously generated functional network is thought to form the initial framework for the emergent cortical representation of stimulus orientation since visual stimulation at or before eye opening drives weakly orientation-selective responses at the cellular and modular scale \cite{chang2020experience, chapman1996development, crair1998role, godecke1997development, schmidt1999matching} and spontaneous activity prior to eye opening is predictive of the representations of stimulus orientations at eye opening \cite{smith2018distributed}. %TODO: figure of visual cortex: retina, LGN and V1. 
	
	The ferret is a species with a well-defined modular network of orientation-selective responses. Newborn ferrets open their eyes around thirty days after they were born. More than two days before eye-opening, the visual cortex was assumed to be visually naive. At least four days after eye-opening, the cortical network can gather information environment to become experienced. The data was collected at different time points from visually naive and experienced visual cortex \cite[Figure 1a]{tragenap2023nature}. In visually naive animals, their network responses are strong but highly variable, while in visually experienced animals, the diversity of responses was reduced and responses became more reliable. 
	% TODO: figure eye-opening time line (according to the figure 1 a manuscript)
	
	% Four activity properties for evaluation? Perhaps not realy necessary since more details in the method.
	% Feedforward recurrent alignment/feedforward recurrent hypothesis introduction. Meaning and idea of it.
	To explore the underlying mechanism that builds reliable network responses, authors in the work from the lab of M.Kaschube and D.Fitzpatrick \cite{tragenap2023nature} developed the "feedforward recurrent alignment hypothesis". The hypothesis proposed that the initial evoked activity pattern reflect novel visual input that is only poorly aligned with the endogenous networks and that highly reliable visual representations emerge from a realignment of feedforward and recurrent networks that is optimal for amplifying these novel patterns of visually driven activity. 
	
	\subsection{Theoretical Exploration and Extensions}
	% Symmetric interaction networks. Problems.
	Ample computational work suggests that recurrent connections can give rise to amplification within subnetworks of coactive network units \cite{abbott1994decoding, ben1995theory, douglas1995recurrent, miller2016canonical, christie2017cortical, peron2020recurrent}. Input that aligns more with such a subnetwork is expected to elicit a more robust response. 
	Therefore, the "feedforward recurrent alignment hypothesis" was developed based on a conceptual computational network model of early cortex and its response to visually evoked input using a minimal linear recurrent network \cite{tragenap2023nature}. Each unit represents the pooled activity in a local group of neurons. Connections between units describe the net interactions between local pools. For simplicity, it was assumed that the net interactions are symmetric, resulting symmetric interaction matrix for recurrent network. They found out that the differences in the degree of feedforward recurrent alignment could reproduce the characteristics of network behavior from experimental observations that distinguish naive and experienced visual cortex evoked responses (Method section \ref{sec:ffrec_sym} and Results section \ref{sec:results_symmetric}). 
	% The four respononse properties here? -- Actually also well introduced in methods. 
	
	% Involved extensions: asymmetric, low rank matrices, black box and Hebb.
	%TODO: structure figure for extensions and together for work structure.
	To explore the potential of feedforward recurrent alignment hypothesis, we develop some theoretical explorations and extensions in this work on the existing feedforward recurrent alignment model. The explorations cover the perspectives of different recurrent network structure, experimental usability, and plasticity. 
	
	% asymmetric interaction network and related sections.
	First part of exploration is to adapt the feedforward recurrent alignment modeling on asymmetric net interactions. The previous model considered for simplicity symmetric net interaction, which however lose the biological generality of cortical network structure. Asymmetric networks represent more general cortical network structure, but can raise more complicated dynamics and result patterns in complex plane. To solve this problem, we try out different modifications to adapt the prior feedforward recurrent alignment modeling. At the same time, the modified model should still reflect the experimental observations from data gather in the earlier work from the lab of of M.Kaschube and D.Fitzpatrick \cite{tragenap2023nature}. The detailed method for this part is introduced in section \ref{sec:ffrec_asym} and corresponding results in section \ref{sec:asymmetric_results}. 
	
	% Low rank network and related sections. 
	Second part focus on another recurrent network structure suggested by \cite{dubreuil2022role, beiran2021shaping, mastrogiuseppe2018linking}. The authors mentioned that large-scale neural recording have established that the transformation of sensory stimuli into motor outputs relies on low-dimensional dynamics at the population level, while individual neurons exhibit complex selectivity. Prioir experiments in behaving animals  have found that trajectories of neural activity are typically restricted to low-dimensional manifolds in that space \cite{machens2010functional, mante2013context,  rigotti2013importance, gao2015simplicity, gallego2018cortical,  chaisangmongkon2017computing,  wang2018flexible,  sohn2019bayesian}.
	Besides, they introduced the class of models, low-rank recurrent networks, directly embodies the idea of low-dimensional collective dynamics, opens the door to relating connectivity and dynamics, and provides a framework that unifies a number of specific RNN classes \cite{mastrogiuseppe2018linking}. Low-rank RNNs rely on connectivity matrices that are restricted to be low-rank, which directly generate low-dimensional dynamics. 
	Therefore, we also interested in the adaptation of feedforward recurrent alignment on the promising low-rank recurrent network. Methods for construction of low-rank recurrent networks are introduced in section \ref{sec:low_rank_method} and its corresponding results in section \ref{sec:low_rank_result}. 
	
	% Black Box model and related sections.
	The third part is related to the perspective of experimental usability of feedforward recurrent alignment. Since the whole cortical network structure is difficult to access during laboratory experiments, the degree of alignment between feedforward inputs and recurrent network structure is also except accessibility. Therefore, it could be helpful if the feedforward recurrent alignment model could be reformulated only demanding experimental measurable information. The work by \cite{marre2009reliable} pointed out that the reliability of evoked dynamics in recurrent networks is dependent on the stimulus used. \cite{mulholland2023selective} then further gave the prediction that stimulus inputs that aligned with the structure of ndogenous subnetworks would be recurrently amplified, leading to more reliable evoked responses and constraining the potential outputs of the network. Based on an experimental method introduced by the lab of H.Mulholland \cite{cosyne2023}, white noise evoked activity can be used to generate spontaneous-like activity patterns. Thus, inspired from the series of work, we explore the possibility of modifying the feedforward recurrent alignment with white noise evoked activity. Since the original recurrent network structure is assumed to be unknown, we call it the black box recurrent network model. The method for this part is described in section \ref{sec:black_box_method} and the results in section \ref{sec:black_box_result}. 
	
	% Hebbian Learning and related sections.
	Finally, the last part regards the feedforward recurrent alignment hypothesis from the perspective of network learning and plasticity.  Connection strengths can be modified by learning from experience, and the degree of learning from each experience is a parameter that can be modified. The simplest kind of learning is Hebbian learning (Hebb, 1949), where the weight between a sending and a receiving node increases if the two nodes are active at the same time. In other words "Nodes that fire together, wire together." This enables learning the correlational structure of the environment \cite{read2021neural}. Thus, including learning rule in the feedforward recurrent network could also be a potential perspective to explore the mechanism for the experience-driven change of response reliability. The methods the exploration of feedforward recurrent alignment considering simple Hebbian rule are introduced in section \ref{sec:Hebb_learn} and the results for it in section \ref{sec:Habb_result}. 
	
	%\subsubsection{Evolved Extensions}
		
	%\subsection{Work Structure}

	%The clarifying of mean firing rate in introduction perhaps?
\end{document}